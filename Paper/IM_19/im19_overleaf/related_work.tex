%Traffic monitoring can be classified into two main categories. The first one is traffic matrix estimation which usually estimates the missing traffic data, and the second one is traffic prediction which forecasts the future traffic. 
Traffic matrix estimation and prediction, especially in backbone networks, have attracted a lot of attention in the research community. 
To solve the traffic matrix estimation problem, the authors in \cite{hu2015cracking} exploited the compressive sensing and network tomography to recover the missing data in the traffic matrices. Moreover, the studies \cite{xie2016accurate} and \cite{xie2017accurate} considered the spatial and temporal features by proposing a new structure of the traffic matrices (3-way tensor).
%in order to estimate the missing elements in the traffic measurement data. 
The above methods can reduce the monitoring overhead by using only a small number of measurement data (under 30$\%$). However, these approaches focused on the traffic interpolation problem while many network applications and management tasks require to forecast the future traffic usage or demand.

In the traffic matrix prediction problem, originally, researchers referred to some simple statistical models such as ARIMA or Gaussian model \cite{zhou2017accurate}. However, such simple models cannot handle the complexity and dynamics of the communication behavior in modern networks. 
%to handle the complicated and dynamic in the communication behavior, 
To this end, deep learning techniques have been exploited more in predicting traffic \cite{wang2017spatiotemporal, cao2018interactive, nie2016traffic}. 
In \cite{wang2017spatiotemporal} and \cite{cao2018interactive}, the authors utilized the Long Short-Term Memory and Convolutional Neural Network for capturing the spatiotemporal feature and predicting the network traffic in data center and cellular networks, respectively. In backbone network, Nie et al. \cite{nie2016traffic} used Restricted Boltzmann Machine to capture the dynamic features of network. The authors then proposed two separated deep belief network models for solving the traffic estimation and prediction, independently. More recent, the results in \cite{azzouni2018neutm} and \cite{vinayakumar2017applying} showed the superiority of LSTM network in modeling the temporal feature and the long-range dependencies of network traffic. Unfortunately, all the approaches proposed so far assumed that the input contains only the ground-truth data. To the best of our knowledge, there is no existing work addressing the problem of traffic prediction under missing ground-truth data in backbone network.
